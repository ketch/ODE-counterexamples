\documentclass{article}
\usepackage[utf8]{inputenc}
\usepackage{amsmath}
\usepackage{amsthm}
\usepackage{amssymb}

\newtheorem{example}{Example}
\newtheorem{theorem}{Theorem}

\title{Conditions for conservative dynamical systems}
\author{David I. Ketcheson}
\date{October 2019}

\begin{document}

\maketitle

Consider the initial value problem
\begin{align} \label{IVP}
\begin{split}
    u'(t) & = f(t,u) \\
    u(t_0) & = u_0.
\end{split}
\end{align}

We are interested in conditions such that
\begin{align} \label{conservation}
\|u(t)\| = \|u_0\| \text{ for all } t, u_0.
\end{align}

\begin{theorem}
Let $f(t,y)$ be a time-independent, normal, linear operator
(i.e. $f(t,y)=Ly$ with $LL^T=L^T L$).  Then \eqref{conservation} holds
if and only if the spectrum of the jacobian of $f$
is purely imaginary.
\end{theorem}

One is immediately led to ask whether (either part of) the assertion in the theorem holds when $f$ is some other type of operator.  Some counterexamples follow; they have appeared elsewhere.

\begin{example}
Consider the system
\begin{subequations} \label{nonlinear-ode}
\begin{align}
u_1'(t) & = \frac{-u_2}{u_1^2 + u_2^2} \\
u_2'(t) & = \frac{ u_1}{u_1^2 + u_2^2}.
\end{align}
\end{subequations}
Condition \eqref{conservation} holds but the spectrum of $\nabla_u f(u)$
is real, consisting of one positive and one negative eigenvalue.
\end{example}

\begin{example}
This example is taken from Dekker \& Verwer, who developed it based on earlier sources
going back to Vinograd.  Let $f$ be a linear, time-dependent operator given by $A(t) = E(t) D E^{-1}(t)$ where $D=diag(i,-i)$ and
\begin{align*}
E(t) = \begin{pmatrix}
\cos(\omega t - \theta) & -\cos(\omega t) \\
-\sin(\omega t - \theta) & \sin(\omega t)
\end{pmatrix}.
\end{align*}
Since $A(t)$ is similar to $D$, its spectrum is purely imaginary.  But, taking
$\omega=1$ and $\theta=\pi/4$ for example, the solution of \eqref{IVP} grows exponentially in time with
exponent equal to the real part of $\sqrt{-2+2i}$.
\end{example}

\end{document}

